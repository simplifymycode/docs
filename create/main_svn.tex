%%%%%%%%%%%%%%%%%%%%%%%%%%%%%%%%%%%%%%%%%
% Frequently Asked Questions
% LaTeX Template
% Version 1.0 (22/7/13)
%
% This template has been downloaded from:
% http://www.LaTeXTemplates.com
%
% Original author:
% Adam Glesser (adamglesser@gmail.com)
%
% License:
% CC BY-NC-SA 3.0 (http://creativecommons.org/licenses/by-nc-sa/3.0/)
%
%%%%%%%%%%%%%%%%%%%%%%%%%%%%%%%%%%%%%%%%%

\documentclass[11pt]{article}

\usepackage[margin=1in]{geometry} % Required to make the margins smaller to fit more content on each page
\usepackage[linkcolor=blue]{hyperref} % Required to create hyperlinks to questions from elsewhere in the document
\hypersetup{pdfborder={0 0 0}, colorlinks=true, urlcolor=blue} % Specify a color for hyperlinks
\usepackage{todonotes} % Required for the boxes that questions appear in
\usepackage{tocloft} % Required to give customize the table of contents to display questions
\usepackage{microtype} % Slightly tweak font spacing for aesthetics
\usepackage{palatino} % Use the Palatino font

\setlength\parindent{0pt} % Removes all indentation from paragraphs

% Create and define the list of questions
\newlistof{questions}{faq}{\large List of Frequently Asked Questions} % This creates a new table of contents-like environment that will output a file with extension .faq
\setlength\cftbeforefaqtitleskip{4em} % Adjusts the vertical space between the title and subtitle
\setlength\cftafterfaqtitleskip{1em} % Adjusts the vertical space between the subtitle and the first question
\setlength\cftparskip{.3em} % Adjusts the vertical space between questions in the list of questions

% Create the command used for questions
\newcommand{\question}[1] % This is what you will use to create a new question
{
\refstepcounter{questions} % Increases the questions counter, this can be referenced anywhere with \thequestions
\par\noindent % Creates a new unindented paragraph
\phantomsection % Needed for hyperref compatibility with the \addcontensline command
\addcontentsline{faq}{questions}{#1} % Adds the question to the list of questions
%\todo[inline, color=green!40]{\textbf{#1}} % Uses the todonotes package to create a fancy box to put the question
\todo[inline, color=gray!40]{\textbf{#1}} % Uses the todonotes package to create a fancy box to put the question
\vspace{1em} % White space after the question before the start of the answer
}

% Uncomment the line below to get rid of the trailing dots in the table of contents
%\renewcommand{\cftdot}{}

% Uncomment the two lines below to get rid of the numbers in the table of contents
%\let\Contentsline\contentsline
%\renewcommand\contentsline[3]{\Contentsline{#1}{#2}{}}

\begin{document}

%----------------------------------------------------------------------------------------
%	TITLE AND LIST OF QUESTIONS
%----------------------------------------------------------------------------------------

\begin{center}
\Huge{\bf \emph{FAQ - new svn repository and project}} % Main title
\end{center}

\listofquestions % This prints the subtitle and a list of all of your questions

\newpage % Comment this if you would like your questions and answers to start immediately after table of questions

%----------------------------------------------------------------------------------------
%	QUESTIONS AND ANSWERS
%----------------------------------------------------------------------------------------

\question{Where do I find a quickstart guide for svn?}\label{new-question}

For quickstart of svn you may refer to:
\begin{verbatim}
http://svnbook.red-bean.com/
nightly/de/svn-book.html\#svn.intro.quickstart
\end{verbatim}

%------------------------------------------------

\question{How do I set up a new svn repository?}\label{new-question}

You may log on as root and choose an appropriate location for you repository:

\begin{verbatim}
su
mkdir /var/svn
\end{verbatim}

If you assign the sticky-bit (t), files can only be deleted by the owner:

\begin{verbatim}
chmod +t /var/svn
\end{verbatim}

Log off as root:
\begin{verbatim}                                                                                                                                                                                                                          
exit
\end{verbatim}

To setup the new repository use the \texttt{svnadmin} command:

\begin{verbatim}
svnadmin create /var/svn/repos ; ls /var/svn/repos
\end{verbatim}

%------------------------------------------------

\question{How do I start a new svn project?}\label{new-question}

Start with a new folder and generate the basic directory structure of your svn project:

\begin{verbatim}
mkdir ~/work
mkdir ~/work/myproject
mkdir ~/work/myproject/trunk
mkdir ~/work/myproject/tags
mkdir ~/work/myproject/branches
\end{verbatim}

Place all files you want to assign to the new project into \texttt{trunk}.

%------------------------------------------------

\question{How do I import my new project into the repository?}\label{new-question}

Use \texttt{svn import} and assign a brief description (type in one line):

\begin{verbatim}
svn import ~/work/myproject file:///var/svn/repos/myproject
   -m "initial import myproject"
\end{verbatim}

%------------------------------------------------

\question{How do I list the content of my repository?}\label{new-question}

Use \texttt{svn list} to browse your repository, e.g.:

\begin{verbatim}
svn list file:///var/svn/repos
svn list file:///var/svn/repos/myproject
svn list file:///var/svn/repos/myproject/trunk
\end{verbatim}

%------------------------------------------------

\question{How do I create a tagged version?}

You create a tagged version from trunk, e.g.:

\begin{verbatim}
svn copy file:///var/svn/repos/myproject/trunk
   file:///var/svn/repos/myproject/tags/Version-1.00
   -m "tagged version 1.0"
\end{verbatim}

%------------------------------------------------

\question{How do I initialize a new user branch?}\label{new-question}

You can initialize a new user branch directly from trunk or from a tagged version, e.g.:

\begin{verbatim}
svn copy file:///var/svn/repos/myproject/trunk
   file:///var/svn/repos/myproject/branches/anyuser
   -m "first user branch"
\end{verbatim}

\begin{verbatim}
svn copy file:///var/svn/repos/myproject/tags/Version-1.00
   file:///var/svn/repos/myproject/branches/anyuser
   -m "first user branch"
\end{verbatim}

%------------------------------------------------

\question{How do I check out a working copy?}

You can check out a working copy from trunk, a user's branch or a tagged version, e.g.

\begin{verbatim}
svn checkout file:///var/svn/repos/myproject/trunk myproject

svn co file:///var/svn/repos/myproject/branch/anyuser myproject

svn co file:///var/svn/repos/myproject/tags/Version-1.00 myproj_v1.0
\end{verbatim}

%------------------------------------------------

\question{How do I check out a working copy of a specific revision No.?}

A working copy of a specific revision number is checked out as follows (e.g. from trunk):

\begin{verbatim}
svn checkout -r 6 file:///var/svn/repos/myproject/trunk myproject
\end{verbatim}

\question{How do I change a given log message of my repository?}

Use \texttt{svnadmin setlog} with \texttt{--bypass-hooks},
e.g. for log message of revision No. 6:

\begin{verbatim}
echo "new log message" > newlogmessage

svnadmin setlog /var/svn/repos -r 6 newlogmessage --bypass-hooks
\end{verbatim}

Please notice: the \texttt{--bypass-hooks} should be used very carefully. \\

Alternatively a hook called \texttt{pre-revprop-change} must be created by your administrator.

In this case you can use the same command withouth the \texttt{--bypass-hooks}:

\begin{verbatim}
svnadmin setlog /var/svn/repos -r 6 newlogmessage
\end{verbatim}

%------------------------------------------------

\question{How do I show the version number of my svn installation?}

You can show the version number of your svn program as follows:

\begin{verbatim}
svn --version --quiet
\end{verbatim}

%------------------------------------------------

\question{How do I show the version number of a given svn repository?}

The version number of a given repository (i.e. svn version used to generate the repository)

is contained in the \texttt{./db/format} file:

\begin{verbatim}
more /var/svn/repos/db/format
\end{verbatim}

The number listed on the first line indicates the version number as follows: \\

%2 =\textgreater \ 1.4 \ ; \  3 =\textgreater \ 1.5 \ ; \ 4 =\textgreater \ 1.6 \ ; \ 5 =\textgreater \ 1.7 \ ; \ 6 =\textgreater \ 1.8
1 =\textgreater \ 1.1+ \ ; \ 2 =\textgreater \ 1.4+ \ ; \  3 =\textgreater \ 1.5+ \ ; \ 4 =\textgreater \ 1.6+ \ ; \ 5 =\textgreater \ 1.7-dev \ ; \ 6 =\textgreater \ 1.8 ; \ 7 =\textgreater \ 1.9 \\

The + symbol indicates upward-compatibility, e.g. 1.4+ says it was generated by svn version

1.4 and is readable by higher versions.

%------------------------------------------------

\question{How do I archive my repository?}

Use \texttt{svnadmin dump} to export the content of your repository to a portable file:

\begin{verbatim}
svnadmin dump /var/svn/repos > svn_repos.dump
\end{verbatim}

You may also want to create a log file of your repository:

\begin{verbatim}
svn log file:///var/svn/repos > svn_repos.log
\end{verbatim}

%------------------------------------------------

\question{How do I set up my repository from archive?}

Use \texttt{svnadmin load} to import the content of your archive file to your repository:

\begin{verbatim}
svnadmin create /var/svn/repos_new

svnadmin load /var/svn/repos_new < svn_repos.dump
\end{verbatim}




%----------------------------------------------------------------------------------------

\end{document}
