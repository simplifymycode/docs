%%%%%%%%%%%%%%%%%%%%%%%%%%%%%%%%%%%%%%%%%
% Frequently Asked Questions
% LaTeX Template
% Version 1.0 (22/7/13)
%
% This template has been downloaded from:
% http://www.LaTeXTemplates.com
%
% Original author:
% Adam Glesser (adamglesser@gmail.com)
%
% License:
% CC BY-NC-SA 3.0 (http://creativecommons.org/licenses/by-nc-sa/3.0/)
%
%%%%%%%%%%%%%%%%%%%%%%%%%%%%%%%%%%%%%%%%%

\documentclass[11pt]{article}

\usepackage[margin=1in]{geometry} % Required to make the margins smaller to fit more content on each page
\usepackage[linkcolor=blue]{hyperref} % Required to create hyperlinks to questions from elsewhere in the document
\hypersetup{pdfborder={0 0 0}, colorlinks=true, urlcolor=blue} % Specify a color for hyperlinks
\usepackage{todonotes} % Required for the boxes that questions appear in
\usepackage{tocloft} % Required to give customize the table of contents to display questions
\usepackage{microtype} % Slightly tweak font spacing for aesthetics
\usepackage{palatino} % Use the Palatino font

\setlength\parindent{0pt} % Removes all indentation from paragraphs

% Create and define the list of questions
\newlistof{questions}{faq}{\large List of Frequently Asked Questions} % This creates a new table of contents-like environment that will output a file with extension .faq
\setlength\cftbeforefaqtitleskip{4em} % Adjusts the vertical space between the title and subtitle
\setlength\cftafterfaqtitleskip{1em} % Adjusts the vertical space between the subtitle and the first question
\setlength\cftparskip{.3em} % Adjusts the vertical space between questions in the list of questions

% Create the command used for questions
\newcommand{\question}[1] % This is what you will use to create a new question
{
\refstepcounter{questions} % Increases the questions counter, this can be referenced anywhere with \thequestions
\par\noindent % Creates a new unindented paragraph
\phantomsection % Needed for hyperref compatibility with the \addcontensline command
\addcontentsline{faq}{questions}{#1} % Adds the question to the list of questions
%\todo[inline, color=green!40]{\textbf{#1}} % Uses the todonotes package to create a fancy box to put the question
\todo[inline, color=gray!40]{\textbf{#1}} % Uses the todonotes package to create a fancy box to put the question
\vspace{1em} % White space after the question before the start of the answer
}

% Uncomment the line below to get rid of the trailing dots in the table of contents
%\renewcommand{\cftdot}{}

% Uncomment the two lines below to get rid of the numbers in the table of contents
%\let\Contentsline\contentsline
%\renewcommand\contentsline[3]{\Contentsline{#1}{#2}{}}

\begin{document}

%----------------------------------------------------------------------------------------
%	TITLE AND LIST OF QUESTIONS
%----------------------------------------------------------------------------------------

\begin{center}
\Huge{\bf \emph{FAQ - start with python}} % Main title
\end{center}

\listofquestions % This prints the subtitle and a list of all of your questions

\newpage % Comment this if you would like your questions and answers to start immediately after table of questions

%----------------------------------------------------------------------------------------
%	QUESTIONS AND ANSWERS
%----------------------------------------------------------------------------------------

\question{How do I install additional python modules?}\label{new-question}

You can create and use a virtual environment. More information on this topic is available under the following link:

\begin{verbatim}
https://docs.python.org/3/tutorial/venv.html
\end{verbatim}

%------------------------------------------------

\question{How do I create a virtual environment?}\label{new-question}

Use \texttt{python3} to create a virtual environment in \texttt{tutorial-env}:

\begin{verbatim}
python3 -m venv tutorial-env
\end{verbatim}

%------------------------------------------------

\question{How do I use a virtual environment?}\label{new-question}

You enable your virtual environment as follows:

\begin{verbatim}
source tutorial-env/bin/activate
\end{verbatim}

%------------------------------------------------

\question{How do I add additional modules to my virtual environment?}\label{new-question}

You can add various modules, e.g.:

\begin{verbatim}
pip install netCDF4
pip install matplotlib
pip install jupyter
\end{verbatim}

To see which module are contained in your virtual environment simply type:

\begin{verbatim}
pip list
\end{verbatim}

%------------------------------------------------



%----------------------------------------------------------------------------------------

\end{document}
